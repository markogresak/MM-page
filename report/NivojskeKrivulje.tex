% To je predloga za poročila o domačih nalogah pri predmetih, katerih
% nosilec je Tomaž Curk. Avtor predloge je Blaž Zupan.
%
% Seveda lahko tudi dodaš kakšen nov, zanimiv in uporaben element,
% ki ga v tej predlogi (še) ni. Več o LaTeX-u izveš na
% spletu, na primer na http://tobi.oetiker.ch/lshort/lshort.pdf.
%
% To predlogo lahko spremeniš v PDF dokument s pomočjo programa
% pdflatex, ki je del standardne instalacije LaTeX programov.

\documentclass[a4paper,11pt]{article}
\usepackage{a4wide}
\usepackage{fullpage}
\usepackage[utf8x]{inputenc}
\usepackage[slovene]{babel}
\selectlanguage{slovene}
\usepackage[toc,page]{appendix}
\usepackage[pdftex]{graphicx} % za slike
\usepackage{setspace}
\usepackage{color}
\definecolor{light-gray}{gray}{0.95}
\usepackage{listings} % za vključevanje kode
\usepackage{hyperref}
\renewcommand{\baselinestretch}{1.2} % za boljšo berljivost večji razmak
\renewcommand{\appendixpagename}{Priloge}

\title{\huge{Nivojske krivulje} \\ \Large{\em{Druga projektna naloga}}}
\author{
  Gašper Andrejc (63130002) \\
  David Bašelj (63130009) \\
  Marko Grešak (63130058) \\
  Matic Repše (63130207)
}
\date{\today}

\begin{document}

\maketitle

\section{Uvod}

Namen projekte naloge je bil prikazati rekonstrukcijo neke funkcije ali naravnih podatkov, prikazanih s koordinatami \textit{(zemljepisna širina, zemljepisna dolžina, nadmorska višina)}, z nivojskimi krivuljami. Izrisovali smo dve obliki, prva kot statični dvodimenzionalni graf njvojnic, na podoben način, kot smo to delali v prvem semestru pri predmetu OMA za funkcije dveh spremenjivk. Drug graf pa je aproksimirana rekonstrukcija v obliki interaktivnega 3D grafa.

\section{Podatki}

Za podatke smo si za umetne podatke izbrali nekaj že znanih funkcij \textbf{(TODO: IMENA)}, pomagali pa smo si z vizualizacijami grafov na platformi WolframAlpha, da smo lažje preverjali, kakšne razultate naj pričakujemo. Za podatke iz narave pa smo najprej poizkusili s podatki \href{http://en.wikipedia.org/wiki/Shuttle_Radar_Topography_Mission}{\underline{Shuttle Radar Topography Mission}} (SRTM), ki jih \href{http://dds.cr.usgs.gov/srtm/version2_1/SRTM3/}{\underline{ponuja Nasa}} in so bili navedeni kot priporočen vir podatkov v poročilu. Vendar pa smo sprva imeli problem z iskanjem vira podatkov, porabili smo namreč nekaj ur, da smo prišli do navedene povezave ter ugotovili, kako brati podatke.
\par
Ker smo začeli dvomiti, da nam bo uspelo najti podatke, ki jih ponuja Nasa, je drug del ekipe začel z raziskovanjem \href{https://developers.google.com/maps/documentation/elevation/}{\underline{Google Maps Elevation API}}, kjer imamo na voljo \textit{2500} zahtev na dan, v vsaki zahtevi pa lahko zahtevamo podatke za do \textit{512} točk. Torej lahko na vsakih 24 ur dobimo višine za \(2500 \cdot 512 = 1,280,000\) geolokacijskih točk in če to še pomnožimo z vsemi člani, lahko dobimo do \textit{5,120,000} točk na dan, kar je bilo dovolj, da smo lahko izračunali in vizualizirali nivojske krivulje za nekaj zanimivh pokrajin.


\end{document}
